%! Author = Gio
%! Date = 18/08/2021

% Preamble
\documentclass[11pt]{article}

% Packages
\usepackage{amsmath}
\usepackage{blindtext}

% Document
\begin{document}

\section{simulated annehaling}
\begin{itemize}
    \item https://www.researchgate.net/publication/250772567
    \item https://www.researchgate.net/publication/2619136_A_Simulated_Annealing_code_for_General_Integer_Linear_Programs
    \item https://paginas.fe.up.pt/~sfeyo/Docs_SFA_Publica_Conferences/SFA_JP_19960101_CCE_The%20Simplex-Simulated%20Annealing.pdf
    \item https://onlinelibrary.wiley.com/doi/epdf/10.1002/atr.1183
    \item https://www.hindawi.com/journals/mpe/2018/6193649/
\end{itemize}

\begin{itemize}
    \item place the number in matrix
    \item calculate the simplex and get set it at best -> = S_best, S_iter
    \item then proceed for N iterations using width of S_iter calculate a new solution for the first round -> S_iter
    \item for the nexts explore the neighbourhood changing some numbers in the matrix of the current iteration solution -> S
    \item if is better than S_iter you are ok
    if follow a random factor you are ok
        substitute the current S_iter with S
        if is better then S_best you are ok
            substitute the current S_best with S
    else resize the solution in a way to backtrack a little
\end{itemize}

    from my perspective at first you are guided by luck then by results
    start = high temperature
        explore
        you are able to accept more even if not better 
        many columns -> far from optima -> you are likely to improve.
            improve -> much probable by temperature to delete columns
            not improve -> low probable of adding columns



    alta temperatura, considera molto l'optimality, poco le colonne: good neighbor = low probabilty (0.1)
    sigmoid sulle colonne   low p=tante
    sigmoid sul n           high p=bad
    temperatura multiplicative factor
    

    alta temperatura, good n, poche colonne = 0.3        | muoviti poco
    alta temperatura, good n, tante colonne = 0.1        | muoviti molto poco
    alta temperatura, bad n, poche colonne = 0.5         | aumenta ma sei ancora all'inizio
    alta temperatura, bad n, tante colonne = 0.01        | lascia stare ci pensa lo swap
    bassa temperatura, good n, poche colonne = 0.5       | vedi te
    bassa temperatura, good n, tante colonne = 0.2       | vedi te
    bassa temperatura, bad n, poche colonne = 0.9        | saltella pure 
    bassa temperatura, bad n, tante colonne = 0.05       | se proprio vuoi tanto siamo alla frutta, comunque ci pensa lo swap

    
    alta tempratura dai molto peso alla distanza dall ottimo: poca distanza hai tipo un 1 di togliere colonne, tanta distanza 0.0
    poca temperatura uguale: la ricerca dell'optimo e' stretta all'inizio si tende a stringere when you are in good spot
                                                    poi si allerga dato che aumento le probabilità di resize ma sempre guidata dal'optimo nel neighbor

    alta temperatura, highdiff optimal = 0.1             | non togliere le colonne altrimenti poi si aggiungono ma piuttosto swappa il neighb
    alta temperatura, lowdiff optimal = 0.8              | togli le colonne you are in a good spot
    
    

high t, low cost, low cols = 0.3  
high t, low cost, high cols = 0.1  
high t, high cost, low cols = 0.5   
high t, high cost, high cols = 0.01  
low t, low cost, low cols = 0.5 
low t, low cost, high cols = 0.2 
low t, high cost, low cols = 0.9  
low t, high cost, high cols = 0.05 


\section{simplex}
\begin{itemize}
    \item https://geekrodion.com/blog/operations/simplex
    \item https://www.dam.brown.edu/people/huiwang/classes/am121/Archive/big_M_121_c.pdf
    \item https://cbom.atozmath.com/CBOM/Simplex.aspx?q=tp&q1=2%603%60MAX%60Z%60x1%2cx2%601%2c1%600%2c1%3b0%2c2%3b4%2c1%60%3e%3d%2c%3e%3d%2c%3e%3d%604500%2c9000%2c16000%60%60D%60false%60false%60false%60true%60false%60false%60false&do=1#tblSolution
    \item http://simplex.tode.cz/en/#steps
    \item http://www.maths.qmul.ac.uk/~ffischer/teaching/opt/notes/notes8.pdf
    \item https://www3.nd.edu/~dgalvin1/30210/30210_F07/presentations/simplex_full.pdf
\end{itemize}

Degeneracy happens when the equations in a tableau do not permit any increment of the selected nonbasic variable,
and it may actually be impossible to increase the objective function z in a single pivot step.

\textbf{forma standard} esempio con min
- per ogni vincolo con >= introduco -surplus_var and artificial end set to = the constraint 
- per ogni vincolo con <= introduco +slack                      end set to = the constraint 
- per ogni vincolo con = introduco artificial


\textbf{big M steps} esempio con min
if any constraints have negative constant on the right side, multiply by -1 (have all right positive)
surplus and artificial for >=
slack for <=
artificial for each =
for each artificial ai add -Mai to the objective function (M large) then transform all M in positive by mult -1
then proceed with the basic variables
- A variable can be selected as basic variable only if it corresponds to a column in the tableau that has exactly one
nonzero element and the nonzero element in the column is not in the same row as the nonzero element in the column on another basic variable.
The rest are non basic. in short the columns of basic variables construct a diagonal matrix.
- If basic variables value (that is the right value) are all positive then it is feasible

\textbf{dual} start with min and go with a max
transpose A
b -> c
c -> b
attention: C previously in vertical obtained the P variable setted to 1 when tou transpose to make it a b row become
            -> 8x1 + 8x2 = P      to      -8x1 -8x2 + P = 0
                (maybe because it was an artificial the other constraints were <= and only a slack where added)
then proceed with basic choose the most negative from last row make rapporto in the colum values, each row value with last column (lc/rv) and take the minimum
make the the value choosen in the column as a 1
subtract to make every other value in the column a 0
stop when the last row is all positive
not more interested in basic non basic as solution but get the variables by name from the initial definition of the problem you asked for
- geekrodion sta minimizzando e utilizza le variabili negative di sinistra (moltiplica tutto per -1 in modo da avere il right side negativo)
        e stoppa quando sono tutte nel right side >= 0


\textbf{two phase} phase 1 search BFS can be unfeasible phase two drop artificial and arrive to BFS can be unbounded
original is maximize put in standard form and solve with a different objective at first
phase 1) always minimization of a summation of artificial variables (but can be converted for ease)
the objective is Z=\sum artificial (minimize)       or      Z=\sum -artificial (maximize)      
we convert to maximization of the artif and rewite as Z + a1 + a2 = 0
the artificial are basic now: need to express the objective entirely in terms of non basic aka subtract/sum to make the basics 0s and diagonal 
then proceed the phase 1 removing the most negative values from the objective constraint (last line) with iterations
    - maximization problem—most negative coefficient in z-row
    - minimization problem—most positive coefficient in z-row
end phase1 -> drop artificial variables and rewrite the original objective --> that could also be added in at start
                (transform it in maximization if necessary)(remember ex. min Z=a1+a2 -> max Z=-a1-a2 -> Z+a1+a2=0)
since you changed the line the basic variables are not diagonal and you have to restore it as previously done
now proceed the phase 2 removing all negative with iterations 

if during phase 1,in maximization Z value become negative (<0) the problem is infeasible (minimization/positive)
if at end of phase 1, artificial variables are basic the solution is degenerate (whatch wikipedia or geekrodion)

\end{document}
